\documentclass{article}
\usepackage{amsmath}
\usepackage{amsfonts}
\usepackage{amssymb}
\usepackage{graphicx}
\usepackage{listings}
\usepackage{color}
\usepackage{hyperref}
\usepackage[a4paper, margin=1in]{geometry}

\definecolor{mygreen}{rgb}{0,0.6,0}
\definecolor{mygray}{rgb}{0.5,0.5,0.5}
\definecolor{mymauve}{rgb}{0.58,0,0.82}

\lstset{ %
  backgroundcolor=\color{white},   
  basicstyle=\footnotesize,        
  breaklines=true,                 
  captionpos=b,                    
  commentstyle=\color{mygreen},    
  escapeinside={\%*}{*)},          
  keywordstyle=\color{blue},       
  stringstyle=\color{mymauve},     
}

\begin{document}
\section*{Ejercicio 1: Aproximación y Análisis de Errores en \(\sin(\pi/8)\)}

Utilizando las herramientas aprendidas en clases, realiza una aproximación de \(\sin(\pi/8)\) asumiendo que no tienes acceso a una calculadora o software avanzado. Usa una aproximación simple para \(\pi\) y realiza el análisis de errores utilizando la siguiente relación:

\[
\text{Error Total} = \hat{f}(\hat{x}) - f(x)
\]

Luego, emplea el truco matemático estándar de sumar y restar la misma cantidad para obtener:

\[
\text{Error Total} = (\hat{f}(\hat{x}) - f(\hat{x})) + (f(\hat{x}) - f(x))
\]

De esta manera, se puede definir:

\[
\text{Error Total} = \text{Error computacional} + \text{Error de datos propagados}
\]

Calcula y analiza estos errores para comprender cómo afectan la precisión de la aproximación.

\section*{Ejercicio 2 : Derivadas Numéricas mediante Expansión en Serie de Taylor}

Mediante la expansión en serie de Taylor, realiza las siguientes tareas:

\begin{enumerate}
    \item[a)] Demuestra la expresión para la \textbf{derivada centrada} de una función \(f(x)\) y analiza su error correspondiente.
    \item[b)] Demuestra la expresión para la \textbf{derivada adelantada} de \(f(x)\) y analiza su error correspondiente.
    \item[c)] Demuestra la expresión para la \textbf{derivada retrasada} de \(f(x)\) y analiza su error correspondiente.
    \item[d)] Calcula el \textbf{valor óptimo de \(h\)} para minimizar el error en cada una de las derivadas anteriores y compara los resultados obtenidos.
\end{enumerate}

\section*{Ejercicio 3: Comparación de Métodos para Diferentes Funciones}

\textbf{Objetivo:} Comparar la precisión de los métodos de derivación numérica para diferentes tipos de funciones.

\begin{enumerate}
    \item[a)] \textbf{Instrucciones:}
    \begin{enumerate}
        \item Implementa las siguientes funciones: \(f(x) = \sin(x)\), \(f(x) = e^x\), \(f(x) = \ln(x)\) (en \(x > 0\)), y \(f(x) = x^2\).
        \item Calcula las derivadas numéricas usando los tres métodos para cada función.
        \item Grafica todas las derivadas y compara con las derivadas exactas de cada función.
    \end{enumerate}
    
    \item[b)] \textbf{Preguntas:}
    \begin{enumerate}
        \item ¿Qué función resulta más difícil de derivar numéricamente?
        \item ¿Qué observaciones puedes hacer acerca de la precisión de cada método en función del tipo de función?
    \end{enumerate}
\end{enumerate}


\end{document}